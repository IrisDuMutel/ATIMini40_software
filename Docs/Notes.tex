\documentclass[a4paper]{article}
\usepackage{fullpage} % Package to use full page
\usepackage{parskip} % Package to tweak paragraph skipping
\usepackage{amsmath}

\usepackage{float}
\usepackage{tikz}
\usepackage{xfrac}
\usepackage[outdir=./Plots]{epstopdf}
\usepackage{pgfplots}
\usepackage{graphicx}    
\usepackage{caption}
\usepackage{mathtools}
\usepackage{comment}
\usepackage{gensymb}
\usepackage{textcomp}
\usepackage{xcolor}
\usepackage{hyperref}

\usepackage{tikz}
\usetikzlibrary{shapes,arrows}
\usetikzlibrary{arrows,calc,positioning}

\tikzset{
	block/.style = {draw, rectangle,
		minimum height=1cm,
		minimum width=2cm},
	input/.style = {coordinate,node distance=1cm},
	output/.style = {coordinate,node distance=4cm},
	arrow/.style={draw, -latex,node distance=2cm},
	pinstyle/.style = {pin edge={latex-, black,node distance=2cm}},
	sum/.style = {draw, circle, node distance=1cm},
}

\usepackage{subcaption}
\setlength{\parindent}{1em}
\graphicspath{{./Images/}}
\title{ATI Mini40 DAQ F/T sensor information and tips}
\author{Iris David Du Mutel}

\begin{document}
	
\maketitle
\tableofcontents
\begin{abstract}
	This document is meant to be a summary of all the relevant information found while working with the ATI Mini40 DAQ F/T sensor. It contain all the information and instructions to operate such sensor in various environments such as LabView, Python and MATLAB using the Keysight 34970A Data Acquisition Unit and the  NI USB 6008 DAQ. 
	
	This document is also a description of the files contained in this project.
\end{abstract}


\section{What kind of sensor is the ATI Mini40 DAQ F/T?}

The transducer is a compact, durable, monolithic structure that converts force and torque into analog strain
gauge signals. 
The force applied to the transducer flexes three symmetrically placed beams using Hooke’s law (from page 16 of the manual \hyperref{https://www.ati-ia.com/app_content/documents/9620-05-DAQ.pdf}{category}{name}{9620-05-DAQ}):
\begin{itemize}
	\item s = E·e
	\item s = Stress applied to the beam (s is proportional to force)
	\item E = Elasticity modulus of the beam
	\item e = Strain applied to the beam
\end{itemize}

Semiconductor strain gauges are attached to the beams and act as strain-sensitive resistors. The resistance of
the strain gauge changes as a function of the applied strain as follows:

\begin{itemize}
	\item $\Delta$R = Sa·Ro·e
	\item $\Delta$R = Change in resistance of strain gauge
	\item Sa = gauge factor of strain gauge
	\item Ro = Resistance of strain gauge unstrained
	\item e = Strain applied to strain gauge
\end{itemize}

\subsection{Load calculation}

From page 18 of the \hyperref{https://www.ati-ia.com/app_content/documents/9620-05-DAQ.pdf}{category}{name}{manual}:

\begin{figure}[h!]
	\centering
	\includegraphics[width=0.9\textwidth]{load_calc.png}
	\label{fig:load_calc}
	\caption{Load calculation process}
\end{figure}

Additionally to this, gain correction factor is only required when a customer amplifier is being used. Refer to page 20 of the \hyperref{https://www.ati-ia.com/app_content/documents/9620-05-DAQ.pdf}{category}{name}{manual} for more information.

\section{Wiring and connecting to a DAQ}


There are two different wiring alternatives for the DAQ version of this sensor:

\begin{itemize}
	\item Differential connections to DAQ (\autoref{fig:diff_wiring}) 
	\item Single-ended connections to DAQ(\autoref{fig:se_wiring}) 
\end{itemize}

\begin{figure}[h!]
	\centering
	\includegraphics[width=0.7\textwidth]{diff_wiring.png}
	\caption{Differential wiring connections to data acquisition system (page 35 from \hyperref{https://www.ati-ia.com/app_content/documents/9620-05-DAQ.pdf}{category}{name}{9620-05-DAQ})}
	\label{fig:diff_wiring}
\end{figure}

\begin{figure}[h!]
	\centering
	\includegraphics[width=0.7\textwidth]{se_wiring.png}
	\caption{Single-ended wiring connections to data acquisition system (page 36 from \hyperref{https://www.ati-ia.com/app_content/documents/9620-05-DAQ.pdf}{category}{name}{9620-05-DAQ})}
	\label{fig:se_wiring}
\end{figure}

A connection from the DAQ F/T's AGnd/AIGnd line to the data acquisition system’s analog input ground or analog ground is required in most cases.
This line allows the return of the small amount of current used by the data acquisition system. Noise can result if this current isn't returned via the AGnd/AIGnd path.
For best noise performance, the cabling from the PS/IFPS connector should be shielded and each strain gauge's signals in a twisted pair. The shielding should be connected to the PS/IFPS connector shell and to the shell of the data acquisition system’s connector. If the data acquisition system has no connector or its connector shell is electrically floating, then the shield at the PS/IPFS connector should be connected to the AGnd/AIGnd signal.

\subsection{Sampling}

For best performance in all applications (page 37 from \hyperref{https://www.ati-ia.com/app_content/documents/9620-05-DAQ.pdf}{category}{name}{9620-05-DAQ}),
the transducer electronics have bandwidth of 5kHz to 10kHz (depending on gain settings). This allows
collection of all transducer frequency content. Note: that to satisfy the Nyquist Theorem, the data needs
to be coupled at a rate greater than twice the highest frequency present, even if data at that frequency
is not preferred. The forces and torques will be sampled at that frequency, not having anything to do with the sampling rate of the data acquisition unit. 

The data acquisition unit on the other hand has a maximum aperture time of 400$\mu$s, meaning that's the smallest amount of time it needs for opening and reading from one channel. This aperture time is equivalent to an integration time of 0.02 PLC. The relationship between this two parameters is the following:

\begin{equation}
	\frac{0.02PLC}{50Hz (instrument power frequency)} = 400\mu s
\end{equation}

Using a differential wiring with 12 channels, we will need to multiply that aperture time by the number of channels:

\begin{equation}
	400\mu s \cdot 12 = 4800\mu s
\end{equation}

An aperture time of 4.8 ms is equivalent to a frequency of around 208 Hz. Rounding the aperture time to 5 ms per data volume (a vector containing one voltage value for each channel), we can obtain 10 measurements every 50 ms. 

\begin{figure}
	\centering
	\includegraphics[width=0.8\textwidth]{sample_rate.pdf}
	\label{fig:sample_rate}
	\caption{Representation of trigger timer of 50 ms and 10 scans of }
\end{figure}
\subsection{Range}
As specified in the ATI \hyperref{https://www.ati-ia.com/products/ft/ft_models.aspx?id=mini40}{category}{name}{site}, the range of the sensor for the calibration US-20-40 is the following defined as the average of the worst and best case
scenarios:

\begin{table}[h!]
	\centering
	\caption{Range values in imperial and metric systems\label{tab:range}}
	\begin{tabular}{||c | c | c | c||} 
		\hline
		Fx, Fy & Fz & Tx,Ty & Tz \\ [0.5ex] 
		\hline\hline
		$\pm$ 20 lbf & $\pm$ 60 lbf & $\pm$ 40 lbf-in & $\pm$ 40 lbf-in\\ 
		\hline
		$\pm$ 88.9644 N & $\pm$ 266.893 N & $\pm$ 4.51939 N-m & $\pm$ 4.51939 N-m \\
		\hline
		
	\end{tabular}
\end{table}

\subsection{Resolution}
As specified in the ATI \hyperref{https://www.ati-ia.com/products/ft/ft_models.aspx?id=mini40}{category}{name}{site}, the resolution of the sensor for the calibration US-20-40 is the following defined as the average of the worst and best case
scenarios:

\begin{table}[h!]
	\centering
	\caption{Resolution values in imperial and metric systems\label{tab:resolution}}
	\begin{tabular}{||c | c | c | c||} 
		\hline
		Fx, Fy & Fz & Tx,Ty & Tz \\ [0.5ex] 
		\hline\hline
		1/200 lbf & 1/100 lbf & 1/200 lbf-in & 1/200 lbf-in\\ 
		\hline
		0.022241108 N & 0.0444822 N & 0.000564924 N-m & 0.000564924 N-m \\
		\hline
	\end{tabular}
\end{table}

\subsection{Sensitivity and output range and resolution voltages}

From page 54 of the transducer \href{https://www.ati-ia.com/app_content/documents/9620-05-Transducer%20Section.pdf}{manual}, we can obtain the analog $\pm$ 10 V sensitivity. Using the data from tables \ref{tab:range} and \ref{tab:resolution}, we can obtain the following table:

\begin{table}[h!]
	\centering
	\caption{Sensitivity, range and resolution outputs voltages\label{tab:sensit} (imperial system)}
	\begin{tabular}{||c | c | c | c ||} 
		\hline
		 & Fx, Fy & Fz & Tx,Ty,Tz \\ [0.5ex] 
		\hline\hline
		Analog $\pm$ 10V sensitivity & 2 lbf/V & 6 lbf/V & 4 lbf-in/V\\ 
		\hline
		Range [V] & $\pm$ 10 & $\pm$ 10 & $\pm$ 10 \\
		\hline
		Resolution [V] & 1/400 & 1/600 & 1/800 \\
		\hline
	\end{tabular}
\end{table}

The minimum voltage that the DAQ must be able to measure is 1/800 V = 1.25 mV. The range must be $\pm$ 10 V.

Just for clarity purposes, \autoref{tab:sensit} in metric system would be as follows:

\begin{table}[h!]
	\centering
	\caption{Sensitivity, range and resolution outputs voltages\label{tab:sensit2} (metric system)}
	\begin{tabular}{||c | c | c | c ||} 
		\hline
		& Fx, Fy & Fz & Tx,Ty,Tz \\ [0.5ex] 
		\hline\hline
		Analog $\pm$ 10V sensitivity & 8.89644 N/V & 26.6893 N/V & 17.7929 N/V\\ 
		\hline
		Range [V] & $\pm$ 10 & $\pm$ 10 & $\pm$ 10 \\
		\hline
		Resolution [V] & 1/400 & 1/600 & 1/800 \\
		\hline
	\end{tabular}
\end{table}

\section{Keysight 34970A connection to PC}
The connection is made via a GPIB‑USB‑HS cable. The GPIB‑USB‑HS is an IEEE 488 controller device for computers with USB slots. The GPIB‑USB‑HS achieves maximum IEEE 488.2 performance. The exact model can be found in \hyperref{https://www.amazon.com/Kanonaki-GPIB-USB-HS-Interface-Adapter-Controller/dp/B07Q84XJJF}{category}{name}{Amazon}. The differences with the \hyperref{https://www.newark.com/ni/780570-01/gpib-usb-hs-gpib-control-device/dp/14AJ5119}{category}{name}{original} true version of this device are not the sxope of this document. 

There are various manuals for this DAQ. The most helpful one containing command examples is the \href{https://documentation.help/Keysight-34970A-34972A/documentation.pdf}{Keysight 34970A/34972A Command Reference Manual}. From this manual, the information from the following section was found.

\subsection{Important commands}

\textbf{ROUT:SCAN} : This command selects the channels to be included in the scan list. This command is used in conjunction with the CONFigure commands to set up an automated scan. To start the scan, use the INITiate or READ? command.

\textbf{INStrument:DMM} : This command disables or enables the internal digital multimeter. When you change the state of the internal DMM, the instrument issues a Factory Reset (*RST command).

\textbf{TRIGger:SOURce} : Select the trigger source to control the onset of each sweep through the scan list (a sweep is one pass through the scan list). The instrument will accept a software (bus) command, an immediate (continuous) scan trigger, an external TTL trigger pulse, an alarm-initiated action, or an internally paced timer. Usually used: TIMer = Internally paced timer trigger.

\textbf{TRIGger:TIMer} : This command sets the trigger-to-trigger interval (in seconds) for measurements on the channels in the present scan list. This command defines the time from the start of one trigger to the start of the next trigger, up to the specified trigger count (see TRIGger:COUNt command). A number from 0 seconds to 359,999 with 1 ms resolution. Note that 359,999 seconds is one second less than one hundred hours.


\begin{figure}[h!]
	\centering
	\includegraphics[width=0.7\textwidth]{trigger_timer.png}
	\label{fig:trigger_timer}
	\caption{Trigger timer}
\end{figure}

\begin{figure}[h!]
	\centering
	\includegraphics[width=0.7\textwidth]{trigger_timer2.png}
	\label{fig:trigger_timer2}
	\caption{Trigger timer from hewlett packard manual}
\end{figure}



\textbf{TRIGger:COUNt}: This command specifies the number of times to sweep through the scan list. A sweep is one pass through the scan list. The scan stops when the number of specified sweeps has occurred. An integer from 1 to 50,000 triggers, or continuous (INFinity).

\textbf{INITiate} : This command changes the state of the triggering system from the ``idle" state to the ``wait-for-trigger" state. Scanning will begin when the specified trigger conditions are satisfied following the receipt of the INITiate command. Readings are stored in the instrument's internal reading memory. Note that the INITiate command also clears the previous set of readings from memory.
If a scan list is currently defined (see ROUTe:SCAN command), the INITiate command performs a scan of the specified channels.
Storing readings in memory using the INITiate command is generally faster than sending readings to memory using the READ? command. The INITiate command is also an "overlapped" command. This means that after executing the INITiate command, you can send other commands that do not affect the measurements.
You can store up to 50,000 readings in memory and all readings are automatically time stamped. If memory overflows, the new readings will overwrite the first (oldest) readings stored; the most recent readings are always preserved.
For scanning measurements using the multiplexer modules, an
error is generated if the internal DMM is disabled.
To retrieve the readings from memory, use the FETCh? command.
The readings are not erased from memory when you read them.
You can send the command multiple times to retrieve the same
data in reading memory.

\textbf{FETCh}: This command transfers readings stored in non-volatile memory to the instrument's output buffer, where you can read them into your computer. The readings stored in memory are not erased when you read them with FETCh?.

\textbf{VOLT:DC:APERTURE} : This command enables the aperture mode and sets the integration time in seconds (called aperture time) for DC voltage measurements on the specified channels.


\section{LabView}

For LabView, a calibration file has to be loaded inside the VI. Use the file FT17838.cal and choose desired units. May not work well when using single-ended connection. In that case, use the matrices provided in ATImatrices.m under the MATLAB/Scripts folder.
 
\subsection{Keysight 34970A}

LabView offers two main ways of interacting with the Keysight 34970A DAQ:

\begin{itemize}
	\item General purpose Virtual Instrument Software Architecture (VISA) blocks.NI-VISA is an API that provides a programming interface to control Ethernet/LXI, GPIB, serial, USB, PXI, and VXI instruments in NI application development environments like LabVIEW, LabWindows/CVI, and Measurement Studio. The API is installed through the NI-VISA driver \cite{NIVISA}.
	\item Agilent Technologies / Keysight Technologies 34970A \hyperref{http://sine.ni.com/apps/utf8/niid_web_display.model_page?p_model_id=5547}{category}{name}{drivers}. These block are based on the VISA blocks but offer a more user-friendly approach to configuring the instrument as well as reading data from it. 
\end{itemize}

The \hyperref{https://github.com/IrisDuMutel/ATIMini40_software/tree/master/LabView}{cat1}{visa}{example}  provided in this repository uses generic VISA blocks. In \autoref{fig:labview_block}, the block diagram of the VI can be seen:

\begin{figure}[!h]
	\centering
	\includegraphics[width=0.9\textwidth]{labview_block.png}
	\caption{LabView block diagram}
	\label{fig:labview_block}
\end{figure}

Inside the while loop, the write and read blocks are interacting with the instrument. Every iteration, the \textit{write} block sends the following commands to the DAQ:

\begin{itemize}
	\item DISP OFF: This command turns off the display of the external instrument. This speeds up the sampling process.
	\item MEASure:VOLTage:DC? 10V, 0.001, (@101:106): The first part of the command 'MEASure:VOLTage:DC$?$` is requesting the measurement of the voltage. The question mark indicates a query command. The two numbers following such query are the \textit{range} and the \textit{resolution}, respectively. There are alternative values for these parameters. See more in pages 211 to 217 from the \hyperref{https://www.manualsbase.com/manual/439362/switch/hp_(hewlett-packard)/hp_34970a/}{category}{name}{manual}.
\end{itemize}

\subsection{NI USB 6008}

Using the VI provided by the \hyperref{https://www.ati-ia.com/products/ft/software/daq_software.aspx}{category}{name}{DAQ software download page} from ATI.


\section{Python}

Using dedicated libraries, different software interfaces have been created for the two alternative DAQ units:

\subsection{Keysight 34790A}

Using \hyperref{https://pyvisa.readthedocs.io/en/latest/index.html}{name}{category}{PyVisa} python libraries.

\subsection{NI USB 6008 DAQ}

Using \hyperref{https://nidaqmx-python.readthedocs.io/en/latest/}{name}{cat}{NI-DAQmx} python libraries.

\begin{figure}[h!]
	\begin{subfigure}{.5\textwidth}
		\centering
		\includegraphics[width=0.7\textwidth,angle=180]{NIUSB6008_ATIMini40Connection.jpg}
		\caption{Single-ended connection with only brown-white reference cable grounded}
		\label{fig:NIUSB6008_ATIMini40Connection}
	\end{subfigure}%
	\begin{subfigure}{.5\textwidth}
		\centering
		\includegraphics[width=.8\linewidth]{Forces_check50Hz.png}
		\caption{Fz test check}
		\label{fig:Forces_check50Hz}
	\end{subfigure}
	\caption{Single-ended connection results with non-connected references}
	\label{fig:test1}
\end{figure} 

\begin{figure}[h!]
	\begin{subfigure}{.5\textwidth}
		\centering
		\includegraphics[width=0.7\textwidth,angle=180]{NIUSB6008_ATIMini40Connection_grounded.jpg}
		\caption{Single-ended connection with all reference cables grounded}
		\label{fig:NIUSB6008_ATIMini40Connection}
	\end{subfigure}%
	\begin{subfigure}{.5\textwidth}
		\centering
		\includegraphics[width=.8\linewidth]{Forces_check50Hz_grounded.png}
		\caption{Fz test check}
		\label{fig:Forces_check50Hz_grounded}
	\end{subfigure}
	\caption{Single-ended connection results with grounded references}
	\label{fig:test2}
\end{figure}

\begin{figure}[h!]
	\centering
	\includegraphics[width=0.8\textwidth]{NIUSB6008_GUI.png}
	\caption{Qt GUI for NI USB 6008}
	\label{fig:NIUSB6008_GUI}
\end{figure}

\section{Resources}

\href{https://documentation.help/Keysight-34970A-34972A/documentation.pdf}{Keysight 34970A/34972A Command Reference Manual}

\bibliographystyle{IEEEtran}
\bibliography{biblio}

\end{document}
